\documentclass[a4paper,14pt]{article}
\pagestyle{empty}
\usepackage{color}
\usepackage{mathtools}
\usepackage{ifxetex}
\ifxetex\usepackage{fontspec}\setmainfont[Ligatures=TeX]{CMU Serif}
\else\usepackage[utf8]{inputenc}\usepackage[T2A]{fontenc}
\fi

\begin{document}
\fontsize{16}{22pt}\selectfont
\setlength{\parindent}{0cm}{\Huge Выполнить бета-редукцию} 
\bigskip

\section*{Дано}
T = (($\lambda$a.($\lambda$b.b b) ($\lambda$b.b b)) b) (($\lambda$c.(c b)) ($\lambda$a.a))
% \bigskip

\section*{Решение}
Введем некоторые обозначения: \newline
S = ($\lambda$a.($\lambda$b.b b) ($\lambda$b.b b)) \newline
R = (($\lambda$c.(c b)) ($\lambda$a.a)) \newline
Тогда  выражение примет вид: T = (S b) R \newline

Упростить S сейчас нельзя. \newline
Однако терм R -- можно: \newline
R = ($\lambda$c.(c b)) ($\lambda$a.a)
    $\xrightarrow[\beta]{}$ (($\lambda$a.a) b) $\xrightarrow[\beta]{}$ b \newline
Получаем, что T = (S b) b


S b = ($\lambda$a.($\lambda$b.b b) ($\lambda$b.b b)) b $\xrightarrow[\alpha]{}$ \newline
        ($\lambda$a.($\lambda$d.d d) ($\lambda$d.d d)) b $\xrightarrow[\beta]{}$ 
        ($\lambda$d.d d) ($\lambda$d.d d) \newline
Получили, что S b -- это расходящийся комбинатор $\Omega. \newline

То есть, T = ($\lambda$d.d d) ($\lambda$d.d d) b = \newline
(($\lambda$d.d d) ($\lambda$d.d d)) b, потому что операция аппликации левоассоциативна.
        
        
\end{document}